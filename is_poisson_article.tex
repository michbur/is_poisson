\documentclass[review]{elsarticle}

\usepackage{lineno,hyperref}
\usepackage{amsfonts}
\usepackage{amsmath}
\usepackage{graphicx}
\usepackage{array,booktabs,tabularx}
\modulolinenumbers[5]

\journal{Journal of \LaTeX\ Templates}

%%%%%%%%%%%%%%%%%%%%%%%
%% Elsevier bibliography styles
%%%%%%%%%%%%%%%%%%%%%%%
%% To change the style, put a % in front of the second line of the current style and
%% remove the % from the second line of the style you would like to use.
%%%%%%%%%%%%%%%%%%%%%%%

%% Numbered
%\bibliographystyle{model1-num-names}

%% Numbered without titles
%\bibliographystyle{model1a-num-names}

%% Harvard
%\bibliographystyle{model2-names.bst}\biboptions{authoryear}

%% Vancouver numbered
%\usepackage{numcompress}\bibliographystyle{model3-num-names}

%% Vancouver name/year
%\usepackage{numcompress}\bibliographystyle{model4-names}\biboptions{authoryear}

%% APA style
%\bibliographystyle{model5-names}\biboptions{authoryear}

%% AMA style
%\usepackage{numcompress}\bibliographystyle{model6-num-names}

%% `Elsevier LaTeX' style
\bibliographystyle{elsarticle-num}
%%%%%%%%%%%%%%%%%%%%%%%

\begin{document}

\begin{frontmatter}

\title{Permutation goodness of fit test for Poisson distribution}
\tnotetext[mytitlenote]{Fully documented templates are available in the elsarticle package on \href{http://www.ctan.org/tex-archive/macros/latex/contrib/elsarticle}{CTAN}.}

%% Group authors per affiliation:
\author{Micha\l{} Burdukiewicz}
\address{Radarweg 29, Amsterdam}
\fntext[myfootnote]{Since 1880.}

\author{Piotr Sobczyk}
\address{Radarweg 29, Amsterdam}
\fntext[myfootnote]{Since 1880.}

\author{Stefan R\"{o}diger}
\address{Radarweg 29, Amsterdam}
\fntext[myfootnote]{Since 1880.}

%% or include affiliations in footnotes:
\author[mymainaddress,mysecondaryaddress]{Elsevier Inc}
\ead[url]{www.elsevier.com}

\author[mysecondaryaddress]{Global Customer Service\corref{mycorrespondingauthor}}
\cortext[mycorrespondingauthor]{Corresponding author}
\ead{support@elsevier.com}

\address[mymainaddress]{1600 John F Kennedy Boulevard, Philadelphia}
\address[mysecondaryaddress]{360 Park Avenue South, New York}

\begin{abstract}
This template helps you to create a properly formatted \LaTeX\ manuscript.
\end{abstract}

\begin{keyword}
permutation dPCR Poisson
\MSC[2010] 00-01\sep  99-00
\end{keyword}

\end{frontmatter}

\linenumbers

\section{Introduction}

The amplification chemistry of absolute quantification in the Digital PCR (dPCR) 
is orchestrated by well established methods such as PCR or isothermal 
amplification \cite{pabinger_survey_2014, morley_digital_2014}. But dPCR relies 
on binary state of positive and negative partitions from a sample. Data fitting 
to a Poisson distribution is an essential step during analysis. The countable 
positive independent integer values make the dPCR analysis interesting and the 
Poisson distribution the obvious choice. The target molecule numbers initially 
in a sample can be determined from the numbers of positive and negative 
partitions by probability analysis.

There exist other \textbf{R} packages for analyse and fit distributions \cite{fitdistrplus_2015}.

\section{Methods}
All functions were implemented in the \textbf{R} statistical computing 
environment as described previously \cite{rodiger_r_2015}. The permutation 
goodness of fit test for Poisson distribution (further called PGoF) is based on 
Chi Square test. The $\hat\lambda$ is calculated from counts of positive and 
negative partitions using following relationship:

\begin{equation}
\hat{\lambda} = - \ln \left(1 - \frac{k}{n} \right)
\end{equation}

According to the MIQE Guidelines for Digital PCR \cite{huggett_digital_2013}, $k$ is number of positive 
partitions and $n$ is a total number of partitions. To perform the permutation 
test, we firstly compute $\hat\lambda_R$ for data vector. Next, from the density 
of Poisson distribution we estimate the chance of having negative partition (no 
template molecules) or positive partition (more than zero template molecules). 
Obtained probabilities are used to perform a Chi Square test and the test 
statistic $\chi_R$ is preserved.

Further, we use the $\hat\lambda_R$ to generate a large number of $n$-long 
sample from Poisson distribution. Each sample is binarized to positive and 
negative partitions and we perform the exactly the same procedure as for the 
real sample: we $\hat\lambda_P$, estimate probabilities of negative and positive 
partitions and finally perform the Chi Square test to obtain test statistic 
$\chi_P$.
The p-value of permutation test is defined as:
\begin{equation}   
\textnormal{p-value} = \frac{N_{\chi_P > \chi_R}}{N}
\end{equation}

where $N_{\chi_P > \chi_R}$ is number of times when $\chi_P$ was more extreme than $\chi_R$.

In case of very low $\hat\lambda_R$, sometimes the $\hat\lambda_P$ may be 
exactly zero. In this case the Chi square test has 0 degrees of freedom. 

\section*{References}

\bibliography{dpcr}

\end{document}