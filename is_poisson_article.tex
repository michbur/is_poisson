\documentclass[review]{elsarticle}

\usepackage{lineno,hyperref}
\usepackage{amsfonts}
\usepackage{amsmath}
\usepackage{graphicx}
\usepackage{array,booktabs,tabularx}
\modulolinenumbers[5]

\journal{Journal of \LaTeX\ Templates}

%%%%%%%%%%%%%%%%%%%%%%%
%% Elsevier bibliography styles
%%%%%%%%%%%%%%%%%%%%%%%
%% To change the style, put a % in front of the second line of the current style and
%% remove the % from the second line of the style you would like to use.
%%%%%%%%%%%%%%%%%%%%%%%

%% Numbered
%\bibliographystyle{model1-num-names}

%% Numbered without titles
%\bibliographystyle{model1a-num-names}

%% Harvard
%\bibliographystyle{model2-names.bst}\biboptions{authoryear}

%% Vancouver numbered
%\usepackage{numcompress}\bibliographystyle{model3-num-names}

%% Vancouver name/year
%\usepackage{numcompress}\bibliographystyle{model4-names}\biboptions{authoryear}

%% APA style
%\bibliographystyle{model5-names}\biboptions{authoryear}

%% AMA style
%\usepackage{numcompress}\bibliographystyle{model6-num-names}

%% `Elsevier LaTeX' style
\bibliographystyle{elsarticle-num}
%%%%%%%%%%%%%%%%%%%%%%%

\begin{document}

\begin{frontmatter}

\title{Permutation goodness of fit test for Poisson distribution}
\tnotetext[mytitlenote]{Fully documented templates are available in the elsarticle package on \href{http://www.ctan.org/tex-archive/macros/latex/contrib/elsarticle}{CTAN}.}

%% Group authors per affiliation:
\author{Elsevier\fnref{myfootnote}}
\address{Radarweg 29, Amsterdam}
\fntext[myfootnote]{Since 1880.}

%% or include affiliations in footnotes:
\author[mymainaddress,mysecondaryaddress]{Elsevier Inc}
\ead[url]{www.elsevier.com}

\author[mysecondaryaddress]{Global Customer Service\corref{mycorrespondingauthor}}
\cortext[mycorrespondingauthor]{Corresponding author}
\ead{support@elsevier.com}

\address[mymainaddress]{1600 John F Kennedy Boulevard, Philadelphia}
\address[mysecondaryaddress]{360 Park Avenue South, New York}

\begin{abstract}
This template helps you to create a properly formatted \LaTeX\ manuscript.
\end{abstract}

\begin{keyword}
\texttt{elsarticle.cls}\sep \LaTeX\sep Elsevier \sep template
\MSC[2010] 00-01\sep  99-00
\end{keyword}

\end{frontmatter}

\linenumbers


\section{Methods}

The permutation goodness of fit test for Poisson distribution (further called PGoF) is based on Chi Square test. The $\hat\lambda$ is calculated from counts of positive and negative partitions using following relationship:

\begin{equation}
\hat{\lambda} = - \ln \left(1 - \frac{k}{n} \right)
\end{equation}

According to the MIQUE Guidelines for Digital PCR, $k$ is number of positive partitions and $n$ is a total number of partitions. To perform the permutation test, we firstly compute $\hat\lambda_R$ for data vector. Next, from the density of Poisson distribution we estimate the chance of having negative partition (no template molecules) or positive partition (more than zero template molecules). Obtained probabilities are used to perform a Chi Square test and the test statistic $\chi_R$ is preserved.

Further, we use the $\hat\lambda_R$ to generate a large number of $n$-long sample from Poisson distribution. Each sample is binarized to positive and negative partitions and we perform the exactly the same procedure as for the real sample: we $\hat\lambda_P$, estimate probabilities of negative and positive partitions and finally perform the Chi Square test to obtain test statistic $\chi_P$.

The p-value of permutation test is defined as:
\begin{equation}   
\textnormal{p-value} = \frac{N_{\chi_P > \chi_R}}{N}
\end{equation}

where $N_{\chi_P > \chi_R}$ is number of times when $\chi_P$ was more extreme than $\chi_R$.

In case of very low $\hat\lambda_R$, sometimes the $\hat\lambda_P$ may be exactly zero. In this case the Chi square test has 0 degrees of freedom. 


\section*{References}

\bibliography{mybibfile}

\end{document}